\documentclass[10pt]{article}
\usepackage[margin=1in, paperwidth=8.5in, paperheight=11in]{geometry}
\usepackage{ifpdf, amsmath, amssymb, comment, color, graphicx, stmaryrd, setspace, enumitem, fancyhdr, wrapfig, textcomp, mathptmx, siunitx, multicol}
\usepackage{hyperref}
\hypersetup{
    colorlinks=true,
    urlcolor=blue,
}

\usepackage{tikz}
\usetikzlibrary{trees}

\setlength{\headheight}{14.5pt}
\newcommand{\del}{\nabla}
\newcommand{\Q}{\mathbb{Q}}
\newcommand{\R}{\mathbb{R}}
\newcommand{\Z}{\mathbb{Z}}
\newcommand{\vu}{\mathbf{u}}
\newcommand{\vv}{\mathbf{v}}
\newcommand{\vw}{\mathbf{w}}
\newcommand{\vi}{\mathbf{i}}
\newcommand{\vj}{\mathbf{j}}
\newcommand{\vk}{\mathbf{k}}
\newcommand{\vn}{\mathbf{n}}
\newcommand{\vr}{\mathbf{r}}
\newcommand{\vs}{\mathbf{s}}
\newcommand{\va}{\mathbf{a}}
\newcommand{\vF}{\mathbf{F}}
\newcommand{\vL}{\mathbf{L}}
\newcommand{\vT}{\mathbf{T}}
\newcommand{\vN}{\mathbf{N}}
\newcommand{\vB}{\mathbf{B}}
\newcommand{\comp}{\operatorname{comp}}
\newcommand{\proj}{\operatorname{proj}}
\newcommand{\orth}{\operatorname{orth}}
\newcommand\dotp[1][.5]{\,\mathbin{\vcenter{\hbox{\scalebox{#1}{$\bullet$}}}}\,}


\newenvironment{red}{\color{red}}{\ignorespacesafterend}
\newcommand{\blue}[1]{\textcolor{blue}{#1}}
\newcommand{\green}[1]{\textcolor{green}{#1}}
\renewcommand{\section}[1]{\begin{center} \textbf{#1} \\\end{center}}
%
\hyphenpenalty=5000
\setlength{\parindent}{0in}
%\oddsidemargin=-.25in
\allowdisplaybreaks
\pagestyle{fancy}
\renewcommand{\headrulewidth}{0pt}
\lhead{MATH 203}
\rhead{Fall 2024}
%\lfoot{}
%\cfoot{}

\begin{document}
%


%\onehalfspacing
\allowdisplaybreaks
%##################################################################
\section{PS\#8 -- Directional derivatives and the gradient - \red{Answer key} }

\begin{enumerate}[leftmargin=0pt]
    
    \item Here's an absolutely classic problem that I think is kind of a rite of passage for students in a multivariate calculus course: the lighthouse problem.
    
    In the lighthouse at Point Gradient, the lamp has been knocked slightly out of vertical, so that the axis is tilted just a little. When the light points east, the beam of light is inclined upward at 5 degrees. When the light points north, the beam of light is inclined upward at 2 degrees.
    
    \begin{enumerate}
        \item The beam of light sweeps out a plane; let's call that plane $f(x, y)$ and say that the lighthouse is at the point $(0, 0)$ What's $f_x(0, 0)$, and what's $f_y(0, 0)$? (Hint: the answers aren't $5^\circ$ and $2^\circ$. Draw a picture and use some trigonometry to figure out the slopes.)
        
        \begin{red}
        Slope is rise over run, yeah? So $f_x(0, 0) = \tan 5^\circ$, and  $f_y(0, 0) = \tan 2^\circ$.
        \end{red}
        \item What's $\del f(0, 0)$? 
        
        \begin{red}
        $\del f(0, 0) = \langle f_x(0, 0), f_y(0, 0) \rangle = \langle \tan 5^\circ, \tan 2^\circ\rangle \approx \langle 0.087, 0.035\rangle$. 
        \end{red}
        \item Looking down from above on a map, in which direction is the light beam pointing when it's most significantly inclined from the horizontal? Explain. 
        
        \begin{red} $\del f = \langle \tan 5^\circ, \tan 2^\circ\rangle$. This is about $21.76^\circ$ north of east.
        
        \end{red}
        \item What is the maximum angle of elevation of the plane of the light beam from horizontal? (Hint: you'll now have to do some inverse trigonometry.) 
        
        \begin{red}$|\del f| = \sqrt{(\tan 5^\circ)^2 + (\tan 2^\circ)^2} \approx \sqrt{0.0089} \approx 0.0942.$ 
            
            The angle is $\arctan(0.0942) \approx 5.381^\circ$ or $0.0939$ radians.
        \end{red}
    \end{enumerate}
    
\end{enumerate}

\end{document}