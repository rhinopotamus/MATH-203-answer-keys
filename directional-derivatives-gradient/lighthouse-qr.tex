\input{../header}

\begin{document}
%


%\onehalfspacing
\allowdisplaybreaks
%##################################################################
\section{PS\#8 -- Directional derivatives and the gradient - \red{Answer key} }

\begin{enumerate}[leftmargin=0pt]
    
    \item Here's an absolutely classic problem that I think is kind of a rite of passage for students in a multivariate calculus course: the lighthouse problem.
    
    In the lighthouse at Point Gradient, the lamp has been knocked slightly out of vertical, so that the axis is tilted just a little. When the light points east, the beam of light is inclined upward at 5 degrees. When the light points north, the beam of light is inclined upward at 2 degrees.
    
    \begin{enumerate}
        \item The beam of light sweeps out a plane; let's call that plane $f(x, y)$ and say that the lighthouse is at the point $(0, 0)$ What's $f_x(0, 0)$, and what's $f_y(0, 0)$? (Hint: the answers aren't $5^\circ$ and $2^\circ$. Draw a picture and use some trigonometry to figure out the slopes.)
        
        \begin{red}
        Slope is rise over run, yeah? So $f_x(0, 0) = \tan 5^\circ$, and  $f_y(0, 0) = \tan 2^\circ$.
        \end{red}
        \item What's $\del f(0, 0)$? 
        
        \begin{red}
        $\del f(0, 0) = \langle f_x(0, 0), f_y(0, 0) \rangle = \langle \tan 5^\circ, \tan 2^\circ\rangle \approx \langle 0.087, 0.035\rangle$. 
        \end{red}
        \item Looking down from above on a map, in which direction is the light beam pointing when it's most significantly inclined from the horizontal? Explain. 
        
        \begin{red} $\del f = \langle \tan 5^\circ, \tan 2^\circ\rangle$. This is about $21.76^\circ$ north of east.
        
        \end{red}
        \item What is the maximum angle of elevation of the plane of the light beam from horizontal? (Hint: you'll now have to do some inverse trigonometry.) 
        
        \begin{red}$|\del f| = \sqrt{(\tan 5^\circ)^2 + (\tan 2^\circ)^2} \approx \sqrt{0.0089} \approx 0.0942.$ 
            
            The angle is $\arctan(0.0942) \approx 5.381^\circ$ or $0.0939$ radians.
        \end{red}
    \end{enumerate}

    \item (Activity 10.7.6 quick reference)
    \begin{enumerate}
        \item Find all critical points in the region. \begin{red}$(2,1)$ is a saddle point. \end{red}
        \item Parameterize the horizontal leg and find critical points.
        \begin{red}
            $x=x$ and $y=0$, so $f(x,y) = f(x,0) = x^2-4x$. Critical point at $(2,0)$; also need to check endpoints $(0,0)$ and $(4,0)$.
        \end{red}
        \item Parameterize the vertical leg and find critical points.
        \begin{red}
            $x=0$ and $y=y$, so $f(0, y) = -3y^2+6y$. Critical point at $(0, 1)$; also need to check endpoints $(0,0)$ and $(0, 4)$.
        \end{red}
        \item Parameterize the hypotenuse and find critical points.
        \begin{red}
            $x=x$ and $y=4-x$, so $f(x, 4-x) = x^2 -3(4-x)^2-4x+6(4-x)$. Critical point at $x=7/2$, so $y=1/2$.
        \end{red}
        \item Find absolute max and absolute min.
        \begin{red}
            \begin{itemize}
                \item $f(2,1) = -1$ -- interior critical point
                \item $f(2,0) = -4$ -- horizontal leg critical point
                \item $\mathbf{f(0,1) = 3}$ -- vertical leg critical point
                \item $f(7/2, 1/2) = 1/2$ -- hypotenuse critical point
                \item $f(0,0) = 0$ -- corner
                \item $f(4,0) = 0$ -- corner
                \item $\mathbf{f(0, 4) = -24}$ -- corner
            \end{itemize}
        \end{red}
    \end{enumerate}
    
\end{enumerate}

\end{document}