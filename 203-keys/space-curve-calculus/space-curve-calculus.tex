\documentclass[10pt]{article}
\usepackage[margin=1in, paperwidth=8.5in, paperheight=11in]{geometry}
\usepackage{ifpdf,amsmath, amssymb, comment, color, graphicx, stmaryrd,setspace,enumitem,tikz, fancyhdr, wrapfig, textcomp, mathptmx, siunitx}

\usepackage{hyperref}
\hypersetup{
    colorlinks=true,
    urlcolor=blue,
}

\usepackage{multicol}

\setlength{\headheight}{14.5pt}
\newcommand{\Q}{\mathbb{Q}}
\newcommand{\R}{\mathbb{R}}
\newcommand{\Z}{\mathbb{Z}}
\newcommand{\va}{\mathbf{a}}
\newcommand{\vu}{\mathbf{u}}
\newcommand{\vv}{\mathbf{v}}
\newcommand{\vw}{\mathbf{w}}
\newcommand{\vi}{\mathbf{i}}
\newcommand{\vj}{\mathbf{j}}
\newcommand{\vk}{\mathbf{k}}
\newcommand{\vn}{\mathbf{n}}
\newcommand{\vr}{\mathbf{r}}
\newcommand{\vs}{\mathbf{s}}
\newcommand{\vF}{\mathbf{F}}
\newcommand{\vL}{\mathbf{L}}
\newcommand{\proj}{\operatorname{proj}}
\newcommand{\orth}{\operatorname{orth}}
\newcommand{\comp}{\operatorname{comp}}
\newcommand\dotp[1][.5]{\,\mathbin{\vcenter{\hbox{\scalebox{#1}{$\bullet$}}}}\,}
	
% Solution text is in red. If you want the solutions to show, remove the \iffalse from the definition of the \red command.
\newenvironment{red}{\color{red}}{\ignorespacesafterend}
\newcommand{\blue}[1]{\textcolor{blue}{#1}}
\newcommand{\green}[1]{\textcolor{green}{#1}}
\renewcommand{\section}[1]{\begin{center} \textbf{#1} \\\end{center}}
%
\hyphenpenalty=5000
\setlength{\parindent}{0in}
%\oddsidemargin=-.25in
\allowdisplaybreaks
\pagestyle{fancy}
\renewcommand{\headrulewidth}{0pt}
\lhead{MATH 203}
\rhead{Fall 2024}
%\lfoot{\copyright\ CLEAR Calculus 2010}
\cfoot{}

\begin{document}
%


%\onehalfspacing
\allowdisplaybreaks
%##################################################################
\section{PS\#4: Calculus with space curves - \red{Answer key} }

\begin{enumerate}[leftmargin=0pt]

\item (\href{https://activecalculus.org/multi/S-9-7-Vector-Valued-Functions-Derivatives.html#Ez_9_7_1}{AC Multi 9.7 Exercise 13}) Compute the derivative of each of the following functions in two different ways: (1) use the rules provided in the theorem stated just after Activity 9.7.3, and (2) rewrite each given function so that it is stated as a single function (either a scalar function or a vector-valued function with three components), and differentiate component-wise. Compare your answers to ensure that they are the same.
\begin{enumerate}
    \item $\displaystyle \vr(t) = \sin(t) \langle 2t, t^2, \arctan(t) \rangle$
    
    \begin{red}
        Using the scalar product rule:
        \begin{align*}
            \frac{d\vr}{dt} &= \left(\dfrac{d}{dt}\sin(t)\right) 
            \langle 2t, t^2, \arctan(t) \rangle
            + \sin(t)
            \left(\dfrac{d}{dt} \langle 2t, t^2, \arctan(t) \rangle\right) \\
            &= \cos(t) \langle 2t, t^2, \arctan(t) \rangle
            + \sin(t) \left\langle 2, 2t, \dfrac{1}{1+t^2} \right\rangle \\
            &= \left\langle 2t\cos(t) + 2\sin(t),
                            t^2\cos(t)+2t\sin(t),
                            \arctan(t) \cos(t) + \dfrac{\sin(t)}{1+t^2} \right\rangle
        \end{align*}
        And rewriting first:
        \begin{align*}
            \vr(t) &= \langle 2t \sin(t) , t^2 \sin(t) , \arctan(t) \sin(t)  \rangle \\
            \frac{d\vr}{dt} &= \left\langle
                    \left(\frac{d}{dt} 2t\right) \sin(t) + 2t \left(\frac{d}{dt}\sin(t)\right),
                    \left(\frac{d}{dt}t^2\right)\sin(t) + t^2 \left(\frac{d}{dt}\sin(t)\right),
                    \left(\frac{d}{dt}\arctan(t)\right) \sin(t) + \arctan(t) \left(\frac{d}{dt} \sin(t) \right)
                \right\rangle \\
                &= \left\langle 2t\cos(t) + 2\sin(t),
                t^2\cos(t)+2t\sin(t),
                \dfrac{\sin(t)}{1+t^2} + \arctan(t) \cos(t) \right\rangle
        \end{align*}
    \end{red}

    \item $\vs(t) = \vr(2^t)$, where $\vr(t) = \langle t+2, \ln(t), 1 \rangle$ \begin{red}
        -- Note that $\vr'(t) = \langle 1, \frac1t, 0 \rangle$
    \end{red}
    
    \begin{red}
        Using the chain rule first:
        \begin{align*}
            \vs'(t) &= \vr'(2^t)\, \frac{d}{dt}2^t \\
            &=\left\langle 1, \dfrac{1}{2^t}, 0\right\rangle
            \, 2^t \ln(2) \\
            &= \left\langle 2^t \ln(2), \ln(2), 0 \right\rangle
        \end{align*}
        And simplifying first:
        \begin{align*}
            \vs(t) &= \vr(2^t) = \langle 2^t + 2, \ln(2^t), 1\rangle 
            = \langle 2^t + 2, t\ln(2), 1\rangle \\
            \vs'(t) &= \langle 2^t \ln(2), \ln(2), 0 \rangle
        \end{align*}
    \end{red}
    \item $\displaystyle \vr(t) = \langle \cos(t), \sin(t), t \rangle \dotp \langle -\sin(t), \cos(t), 1 \rangle$
    
    \begin{red}
        Using the dot product rule:
        \begin{align*}
            r'(t) &= \left(\frac{d}{dt}\langle \cos(t), \sin(t), t \rangle \right)\dotp \langle -\sin(t), \cos(t), 1 \rangle
            + \langle \cos(t), \sin(t), t \rangle \dotp \left(\frac{d}{dt} \langle -\sin(t), \cos(t), 1 \rangle\right) \\
            &= \langle -\sin(t), \cos(t), 1 \rangle \dotp \langle -\sin(t), \cos(t), 1 \rangle 
            + \langle \cos(t), \sin(t), t \rangle \dotp \langle -\cos(t), -\sin(t), 0\rangle \\
            &= \left[ \sin^2(t) + \cos^2(t) + 1 \right]
            + \left[ -\cos^2(t) -\sin^2(t) + 0\right] = 1 + 1 - 1 + 0 = 1 (!!)
        \end{align*}
        And rewriting first:
        \begin{align*}
            r(t) &= cos(t)\cdot(-\sin(t)) + \sin(t)\cdot\cos(t) + t\cdot 1 \\
            r(t) &= t \, (!!) \\
            r'(t) &= 1
        \end{align*}
    \end{red}
    \item $\displaystyle \vr(t) = \langle \cos(t), \sin(t), t \rangle \times \langle -\sin(t), \cos(t), 1 \rangle$
    
    \begin{red}
        Using the cross product rule:
        \begin{align*}
            \vr'(t) &= \left(\dfrac{d}{dt} \langle \cos(t), \sin(t), t \rangle \right)\times 
            \langle -\sin(t), \cos(t), 1 \rangle
            + \langle \cos(t), \sin(t), t \rangle \times 
            \left(\dfrac{d}{dt} \langle -\sin(t), \cos(t), 1 \rangle \right) \\
            &= \langle -\sin(t), \cos(t), 1 \rangle
            \times \langle -\sin(t), \cos(t), 1 \rangle
            + \langle \cos(t), \sin(t), t \rangle
            \times \langle -\cos(t), -\sin(t), 0 \rangle \\
            \intertext{The first two vectors are parallel, so their cross product is $\mathbf{0}$.}
            &= \mathbf{0} + \langle \cos(t), \sin(t), t \rangle
            \times \langle -\cos(t), -\sin(t), 0 \rangle 
            = \langle t\sin(t), -t\cos(t), 0 \rangle \quad\text{ (Thanks, WA!)}
        \end{align*}
        And finding the cross product first:
        \begin{align*}
            \vr(t) &= \langle \sin(t) - t\cos(t), -t\sin(t) - \cos(t), \sin^2(t) + \cos^2(t)\rangle 
            = \langle \sin(t) - t\cos(t), -t\sin(t) - \cos(t), 1 \rangle \\
            \vr'(t) &= \dfrac{d}{dt} \langle \sin(t) - t\cos(t), -t\sin(t) - \cos(t), 1 \rangle \\
            &= \langle \cos(t) - (1 \cos(t) + t(-\sin(t))),
            -(1\sin(t) + t\cos(t)) - (-\sin(t)), 0 \rangle \\
            &= \langle t\sin(t), -t\cos(t), 0 \rangle
        \end{align*}
    \end{red}
\end{enumerate}

\item (\href{https://activecalculus.org/multi/S-9-7-Vector-Valued-Functions-Derivatives.html#Ez_9_7_6}{AC Multi 9.7 Exercise 18}) A central force is one that acts on an object so that the force $\vF$ is parallel to the object's position $\vr$. Since Newton's Second Law says that an object's acceleration is proportional to the force exerted on it, the acceleration $\va$ of an object moving under a central force will be parallel to its position $\vr$. For instance, the Earth's acceleration due to the gravitational force that the sun exerts on the Earth is parallel to the Earth's position vector (see figure in the textbook).
	    
\begin{enumerate}
    \item If an object of mass $m$ is moving under a central force, the angular momentum vector is defined to be $\vL = m\vr \times \vv$. Assuming the mass is constant, show that the angular momentum is constant by showing that $\frac{d\vL}{dt} = \mathbf{0}$.
    
    \begin{red}
    Some stuff to keep track of: 
    \begin{itemize}
        \item $m$ is a constant scalar;
        \item $\vr$ is a variable vector (depends on $t$);
        \item $\vv$ is a variable vector (depends on $t$).
    \end{itemize} 
    Seems like the natural thing to do is to use the product rule to compute $\frac{d\vL}{dt}$:
    \begin{align*}
        \frac{d\vL}{dt} &= \frac{d}{dt}\left[m\vr\times\vv \right]
        = \frac{d}{dt}\left[ m\vr \right] \times \vv + 
        m\vr \times \frac{d}{dt} \left[ \vv \right] \\
        \intertext{Well, the derivative of position is velocity, and the derivative of velocity is acceleration:}
        &= [m\vv] \times \vv + m\vr \times \va \\
        \intertext{Now we're getting somewhere. Remember that the cross product of two parallel vectors is $\mathbf 0$. $m\vv$ is certainly parallel to $\vv$, and we're assuming in the context of the problem that $\va$ is parallel to $\vr$, so it's also parallel to $m\vr$. }
        &= \mathbf 0 + \mathbf 0 = \mathbf 0.
    \end{align*}
    Cool, so that tells us that $\vL$ is a constant vector.
    \end{red}
    
    \item Explain why $\vL \dotp \vr = 0$.
    
    \begin{red} 
    $\vL$ was defined as the cross product of $m\vr$ and $\vv$, so it's perpendicular to both of those. Since $m$ is a scalar, $m\vr$ points in the same direction as $\vr$, so if $\vL$ is perpendicular to $m\vr$, it's also perpendicular to $\vr$. Therefore, their dot product is zero.
    \end{red}
    
    \item Explain why we may conclude that the object is constrained to lie in the plane passing through the origin and perpendicular to $\vL$.
    
    \begin{red}
    The equation $\vL \dotp \vr = 0$ reminds me of the vector equation of a plane, with $\vL$ as the (constant) normal vector and $\vr$ playing the role of $\overrightarrow{PP_0}$. Since $\vr$'s initial point is the origin, we thus have the plane that's perpendicular to $\vL$ and passing through the origin.
    
    (Another way to see this: Certainly all the position vectors are perpendicular to $\vL$. Also, certainly all the position vectors emanate from the origin. Also, we've just found that $\vL$ is constant. So the only way this is going to happen is if all the position vectors lie in the \textbf{same} plane -- specifically, the plane containing the origin and perpendicular to the constant vector $\vL$.)
    \end{red}
\end{enumerate}

\end{enumerate}
	
\end{document}