\documentclass[10pt]{article}
\usepackage[margin=1in, paperwidth=8.5in, paperheight=11in]{geometry}
\usepackage{ifpdf,amsmath, amssymb, comment, color, graphicx, stmaryrd,setspace,enumitem,tikz, fancyhdr, wrapfig, textcomp, units, mathptmx, siunitx}

\setlength{\headheight}{14.5pt}
\newcommand{\Q}{\mathbb{Q}}
\newcommand{\R}{\mathbb{R}}
\newcommand{\Z}{\mathbb{Z}}
\newcommand{\va}{\mathbf{a}}
\newcommand{\vu}{\mathbf{u}}
\newcommand{\vv}{\mathbf{v}}
\newcommand{\vw}{\mathbf{w}}
\newcommand{\vi}{\mathbf{i}}
\newcommand{\vj}{\mathbf{j}}
\newcommand{\vk}{\mathbf{k}}
\newcommand{\vn}{\mathbf{n}}
\newcommand{\vr}{\mathbf{r}}
\newcommand{\vF}{\mathbf{F}}
\newcommand{\vL}{\mathbf{L}}
\newcommand{\proj}{\operatorname{proj}}
\newcommand{\orth}{\operatorname{orth}}
\newcommand{\comp}{\operatorname{comp}}
\newcommand\dotp[1][.5]{\,\mathbin{\vcenter{\hbox{\scalebox{#1}{$\bullet$}}}}\,}
	
% Solution text is in red. If you want the solutions to show, remove the \iffalse from the definition of the \red command.
\newenvironment{red}{\color{red}}{\ignorespacesafterend}
\newcommand{\blue}[1]{\textcolor{blue}{#1}}
\newcommand{\green}[1]{\textcolor{green}{#1}}
\renewcommand{\section}[1]{\begin{center} \textbf{#1} \\\end{center}}
%
\hyphenpenalty=5000
\setlength{\parindent}{0in}
%\oddsidemargin=-.25in
\allowdisplaybreaks
\pagestyle{fancy}
\renewcommand{\headrulewidth}{0pt}
\lhead{MATH 203}
\rhead{Spring 2020}
%\lfoot{\copyright\ CLEAR Calculus 2010}
\cfoot{}

\begin{document}
%


%\onehalfspacing
\allowdisplaybreaks
%##################################################################
\section{PS\#5: Space curves, derivatives, and antiderivatives - \red{Answer key} }

\begin{enumerate}[leftmargin=0pt]

\item (AC Multi 9.6 Exercise 15) For each of the following, describe the effect of the parameter $s$ on the parametric curve $\vr(t)$ in the interval $[0, 2\pi]$.
\begin{enumerate}
    \item $\vr(t) = \langle \cos(t), \sin(t) +s \rangle$
    
    \begin{red}
    Increasing $s$ appears to move the graph in the positive $y$ direction.
    \end{red}
    \item $\vr(t) = \langle \cos(t) - s, \sin(t)\rangle$
    
    \begin{red}
    Increasing $s$ appears to move the graph in the negative $x$ direction.
    \end{red}
    \item $\vr(t) = \langle s \cos(t), \sin(t)\rangle$
    
    \begin{red}
    Increasing $s$ appears to stretch the graph along the $x$-axis.
    \end{red}
    \item $\vr(t) = \langle s \cos(t), s \sin(t)\rangle$
    
    \begin{red}
    Increasing $s$ appears to increase the radius of the circle.
    \end{red}
    \item $\vr(t) = \langle \cos(st), \sin(st)\rangle$
    
    \begin{red}
    Increasing $s$ appears to increase the speed at which the trace point moves around the circle. In fact, the trace point moves around the circle $s$ times in the interval from $[0, 2\pi]$.
    \end{red}
\end{enumerate}

\item If a particle is moving in a straight line (but maybe changing speed), what can you say about its acceleration vector? \\
If a particle is moving at constant speed (but maybe changing direction), what can you say about its acceleration vector? (Hint: $\mathbf{a} = \mathbf{a}_T + \mathbf{a}_N$.)

\begin{red}
	The normal component of the acceleration vector is the component that causes a particle to turn. So, if the particle is moving in a straight line, the normal component of the acceleration vector must be zero -- that is, the acceleration vector must be entirely parallel to the tangent vector. (In other words, writing 
	$\mathbf{a} = \mathbf{a}_T + \mathbf{a}_N$, we must have 
	$\mathbf{a}_N = \mathbf{0}$, so 
	$\mathbf{a} = \mathbf{a}_T$.)

   The tangential component of the acceleration vector is the component that causes a particle to speed up or slow down. Therefore, if the particle is moving at constant speed, the tangential component of the acceleration vector must be zero -- that is, the acceleration vector must be entirely parallel to the normal vector. (In other words, writing 
    $\mathbf{a} = \mathbf{a}_T + \mathbf{a}_N$, we must have 
    $\mathbf{a}_T = \mathbf{0}$, so 
    $\mathbf{a} = \mathbf{a}_N$.)
\end{red}

\item (AC Multi 9.7 Exercise 18) A central force is one that acts on an object so that the force $\vF$ is parallel to the object's position $\vr$. Since Newton's Second Law says that an object's acceleration is proportional to the force exerted on it, the acceleration $\va$ of an object moving under a central force will be parallel to its position $\vr$. For instance, the Earth's acceleration due to the gravitational force that the sun exerts on the Earth is parallel to the Earth's position vector (see figure in the textbook).
	    
\begin{enumerate}
    \item If an object of mass $m$ is moving under a central force, the angular momentum vector is defined to be $\vL = m\vr \times \vv$. Assuming the mass is constant, show that the angular momentum is constant by showing that $\frac{d\vL}{dt} = \mathbf{0}$.
    
    \begin{red}
    Some stuff to keep track of: 
    \begin{itemize}
        \item $m$ is a constant scalar;
        \item $\vr$ is a variable vector (depends on $t$);
        \item $\vv$ is a variable vector (depends on $t$).
    \end{itemize} 
    Seems like the natural thing to do is to use the product rule to compute $\frac{d\vL}{dt}$:
    \begin{align*}
        \frac{d\vL}{dt} &= \frac{d}{dt}\left[m\vr\times\vv \right]
        = \frac{d}{dt}\left[ m\vr \right] \times \vv + 
        m\vr \times \frac{d}{dt} \left[ \vv \right] \\
        \intertext{Well, the derivative of position is velocity, and the derivative of velocity is acceleration:}
        &= [m\vv] \times \vv + m\vr \times \va \\
        \intertext{Now we're getting somewhere. Remember that the cross product of two parallel vectors is $\mathbf 0$. $m\vv$ is certainly parallel to $\vv$, and we're assuming in the context of the problem that $\va$ is parallel to $\vr$, so it's also parallel to $m\vr$. }
        &= \mathbf 0 + \mathbf 0 = \mathbf 0.
    \end{align*}
    Cool, so that tells us that $\vL$ is a constant vector.
    \end{red}
    
    \item Explain why $\vL \dotp \vr = 0$.
    
    \begin{red} 
    $\vL$ was defined as the cross product of $m\vr$ and $\vv$, so it's perpendicular to both of those. Since $m$ is a scalar, $m\vr$ points in the same direction as $\vr$, so if $\vL$ is perpendicular to $m\vr$, it's also perpendicular to $\vr$. Therefore, their dot product is zero.
    \end{red}
    
    \item Explain why we may conclude that the object is constrained to lie in the plane passing through the origin and perpendicular to $\vL$.
    
    \begin{red}
    The equation $\vL \dotp \vr = 0$ reminds me of the vector equation of a plane, with $\vL$ as the (constant) normal vector and $\vr$ playing the role of $\overrightarrow{PP_0}$. Since $\vr$'s initial point is the origin, we thus have the plane that's perpendicular to $\vL$ and passing through the origin.
    
    (Another way to see this: Certainly all the position vectors are perpendicular to $\vL$. Also, certainly all the position vectors emanate from the origin. Also, we've just found that $\vL$ is constant. So the only way this is going to happen is if all the position vectors lie in the \textbf{same} plane -- specifically, the plane containing the origin and perpendicular to the constant vector $\vL$.)
    \end{red}
\end{enumerate}

\end{enumerate}
	
\end{document}