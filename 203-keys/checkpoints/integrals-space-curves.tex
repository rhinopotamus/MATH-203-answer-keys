\documentclass[12pt]{article}
\usepackage[margin=1in, paperwidth=8.5in, paperheight=11in]{geometry}
\usepackage{ifpdf,amsmath, amssymb, comment, color, graphicx, stmaryrd,setspace,tikz, fancyhdr, wrapfig, textcomp, mathptmx, siunitx}

\usepackage[shortlabels]{enumitem}


\usepackage{hyperref}
\hypersetup{
    colorlinks=true,
    urlcolor=blue,
}

\setlength{\headheight}{14.5pt}
\newcommand{\Q}{\mathbb{Q}}
\newcommand{\R}{\mathbb{R}}
\newcommand{\Z}{\mathbb{Z}}
\newcommand{\vu}{\mathbf{u}}
\newcommand{\vv}{\mathbf{v}}
\newcommand{\vw}{\mathbf{w}}
\newcommand{\vi}{\mathbf{i}}
\newcommand{\vj}{\mathbf{j}}
\newcommand{\vk}{\mathbf{k}}
\newcommand{\vn}{\mathbf{n}}
\newcommand{\vr}{\mathbf{r}}
\newcommand{\proj}{\operatorname{proj}}
\newcommand{\orth}{\operatorname{orth}}
\newcommand{\comp}{\operatorname{comp}}
\newcommand\dotp[1][.5]{\,\mathbin{\vcenter{\hbox{\scalebox{#1}{$\bullet$}}}}\,}
	
% Solution text is in red. If you want the solutions to show, remove the \iffalse from the definition of the \red command.
\newenvironment{red}{\color{red}}{\ignorespacesafterend}
\newcommand{\blue}[1]{\textcolor{blue}{#1}}
\newcommand{\green}[1]{\textcolor{green}{#1}}
\renewcommand{\section}[1]{\begin{center} \textbf{#1} \\\end{center}}
%
\hyphenpenalty=5000
\setlength{\parindent}{0in}
%\oddsidemargin=-.25in
\allowdisplaybreaks
\pagestyle{fancy}
\renewcommand{\headrulewidth}{0pt}
\lhead{MATH 203}
\rhead{Fall 2024}
%\lfoot{\copyright\ CLEAR Calculus 2010}
\cfoot{}

\begin{document}
%


%\onehalfspacing
\allowdisplaybreaks
%##################################################################
\section{Checkpoint: Integrals of space curves}

Somehow, you managed to glue a motion tracker to a fly. The motion tracker reports that the velocity of the fly at time $t$ is $\displaystyle\vv(t) = \left\langle \frac{t}{1+t^2}, te^{t^2}, \frac{1}{1+t^2} \right\rangle$; for the sake of argument, let's say that the fly started (at time $t=0$) at the origin $(0,0,0)$.

\begin{enumerate}[(a)]
    \item Find a vector equation for $\vr(t)$, the position of the fly at time $t$.
    \item How far did the fly travel on the interval $0\leq t \leq 1$? \\(Use Wolfram Alpha or whatever to evaluate a nasty integral if necessary.)
\end{enumerate}

	
\end{document}