\input{../header.tex}

\begin{document}
%


%\onehalfspacing
\allowdisplaybreaks
%##################################################################
\section{PS\#9 -- Optimization and Lagrange multipliers - \red{Answer key} }

\begin{enumerate}[leftmargin=0pt]

    \item (\href{https://activecalculus.org/multi/S-10-7-Optimization.html#S-10-7-Optimization-8-18}{AC Multi 10.7 Exercise 18}) If a continuous function $f$ of a \textbf{single variable} has two critical numbers $c_1$ and $c_2$ at which $f$ has relative maximum values, then $f$ must have another critical number $c_3$, because ``it is impossible to have two mountains without some sort of valley in between. The other critical point can be a saddle point (a pass between the mountains) or a local minimum (a true valley).'' (From Calculus in Vector Spaces by Lawrence J. Corwin and Robert H. Szczarb.)
    
    Consider the function $f$ defined by $f(x,y) = 4x^2e^y -2x^4 -e^{4y}.$ (From Ira Rosenholz in the Problems Section of the Mathematics Magazine, Vol. 60 NO. 1, February 1987.) \textbf{Note:} This is a function of two variables, so it doesn't have to follow the rule outlined in the previous paragraph.
    
    Show that $f$ has exactly two critical points, and that $f$ has relative maximum values at each of these critical points. 
    
    \begin{red}
    So we're looking for zeroes of the gradient.
    \begin{align*}
        \del f &= \langle f_x(x, y), f_y(x,y) \rangle \\
        &= \langle 8x e^y - 8x^3, 4x^2 e^y - 4e^{4y} \rangle
        \\
        \intertext{Let's focus on the first component:}
        0 &= 8x e^y - 8x^3 = 8x(e^y - x^2) \\
        \intertext{So, either $x = 0$ or $e^y = x^2$. Note that $x=0$ is not going to make the second component zero, so we can discard it. From the second component,}
        0 &= 4x^2 e^y - 4e^{4y} = 4e^y (x^2 - e^{3y}) \\
        \intertext{So, either $e^y = 0$, which it never does, or $x^2 = e^{3y}$. Therefore, I've got two things that both have to be equal to $x^2$, so they must equal each other:}
        e^y &= e^{3y} \\
        y &= 3y \textrm{, so } y = 0 \\
        \intertext{Therefore, }
        x^2 &= e^0 = 1 \textrm{, so } x = \pm 1
    \end{align*}
    So, we have precisely two critical points: $(1, 0)$ and $(-1, 0)$. Let's use the discriminant to classify them. We need to calculate our second partials:
    \[
    \begin{array}{ll}
       f_{xx} = 8e^y-24x^2  & f_{yx} = 8xe^y \\
       f_{xy} = 8xe^y  & f_{yy} = 4x^2 e^y-16 e^{4y}
    \end{array}
    \]
    At the critical point $(1, 0)$:
    \[
    \begin{array}{ll}
        f_{xx} = 8e^0-24(1)^2 = -16 & f_{yx} = 8(1)e^0 = 8 \\
        f_{xy} = 8(1)e^0 = 8 & f_{yy} = 4(1)^2 e^0 - 16 e^0 = -12
    \end{array}
    \]
    So $D = (-16)(-12) - 8^2 = 128 > 0$, and $f_{xx} < 0$, so this is a maximum.
    
    At the critical point $(-1, 0)$:
    \[
    \begin{array}{ll}
        f_{xx} = 8e^0-24(-1)^2 = -16 & f_{yx} = 8(-1)e^0 = -8 \\
        f_{xy} = 8(1)e^0 = -8 & f_{yy} = 4(1)^2 e^0 - 16 e^0 = -12
    \end{array}
    \]
    So $D = (-16)(-12) - (-8)^2 = 128 > 0$, and $f_{xx} < 0$, so this is a maximum.
    \end{red}
    
    Explain how this function $f$ illustrates that it really is possible to have two mountains without some sort of valley in between. Use appropriate technology to draw the surface defined by $f$ to see graphically how this happens.
    
    \begin{red}
    Yeah, so \href{https://c3d.libretexts.org/CalcPlot3D/index.html?type=z;z=4x^2e^y-2x^4-e^(4y);visible=true;umin=-2;umax=2;vmin=-2;vmax=2;grid=50;format=normal;alpha=-1;constcol=rgb(255,0,0);view=0;contourcolor=red;fixdomain=false&type=window;hsrmode=3;nomidpts=true;anaglyph=-1;center=2.507369696231452,8.657081705234903,4.332208854072867,1;focus=0,0,0,1;up=-0.3279946341240409,-0.34507444582480573,0.8793993102251901,1;transparent=false;alpha=140;twoviews=false;unlinkviews=false;axisextension=0.7;xaxislabel=x;yaxislabel=y;zaxislabel=z;edgeson=true;faceson=true;showbox=true;showaxes=true;showticks=true;perspective=true;centerxpercent=0.5;centerypercent=0.5;rotationsteps=30;autospin=true;xygrid=false;yzgrid=false;xzgrid=false;gridsonbox=true;gridplanes=false;gridcolor=rgb(128,128,128);xmin=-2;xmax=2;ymin=-2;ymax=2;zmin=-2;zmax=2;xscale=1;yscale=1;zscale=1;zcmin=-4;zcmax=4;zoom=0.886667;xscalefactor=1;yscalefactor=1;zscalefactor=1}{check this function out} (that's clickable and takes you to CalcPlot3D):
    
    \begin{center}
        \includegraphics[width=0.75\textwidth]{../images/10-7-18.png}
    \end{center}
    
    So there kind of \textbf{is} a valley between the two mountains, but there's no point at the bottom of the valley, because the valley just continues down forever without ever hitting a critical point.
    \end{red}
    
    \item (\href{https://activecalculus.org/multi/S-10-8-Lagrange-Multipliers.html#S-10-8-Lagrange-Multipliers-6-12}{AC Multi 10.8 Exercise 12}) The Cobb-Douglas production function is used in economics to model production levels based on labor and equipment. Suppose we have a specific Cobb-Douglas function of the form \[f(x, y) = 50 x^{0.4}y^{0.6},\] where $x$ is the dollar amount spent on labor and $y$ the dollar amount spent on equipment. Use the method of Lagrange multipliers to determine how much should be spent on labor and how much on equipment to maximize productivity if we have a total of \$1.5 million dollars to invest in labor and equipment.
    
    \begin{red}
    The method of Lagrange multipliers needs us to have a constraint. Here the constraint is the total amount of money:
    \[x + y = 1.5 \textrm{ million}.\]
    So, our $f$ is the function given in the problem, our $g(x,y) = x+y = 1.5$. Off we go to Lagrange mulitpliers land. We need to calculate the gradients: 
    \begin{align*}
        \del f &= \langle 20 x^{-0.6} y^{0.6}, 30 x^{0.4} y^{-0.4} \rangle \\
        \del g &= \langle 1, 1 \rangle
        \intertext{So, setting up $\del f = \lambda \del g$ together with our constraint:}
        20 x^{-0.6} y^{0.6} &= \lambda\cdot 1 \\
        30 x^{0.4} y^{-0.4} &= \lambda\cdot 1 \\
        x+y &= 1.5
        \intertext{It's nice that we have two things that both equal $\lambda$, so they must just equal each other. Also, our constraint tells us that $y=1.5-x$:}
        20 x^{-0.6} (1.5-x)^{0.6} &= 30 x^{0.4} (1.5-x)^{-0.4} \\
        20\frac{(1.5-x)^{0.6}}{x^{0.6}} &= 30 \frac{x^{0.4}}{(1.5-x)^{0.4}}
        \intertext{Clear denominators. It's so nice that the powers add up to 1, eh?}
        20 (1.5-x)^{0.6} (1.5-x)^{0.4} &= 30 x^{0.4} x^{0.6} \\
        20 (1.5-x) &= 30 x \\
        30 - 20x &= 30x \\
        30 &= 50x \\
        x = 0.6; &\quad y = 0.9
    \end{align*}
    So we should spend $\$600,000$ on labor and $\$900,000$ on equipment.
    \end{red}

\item (\href{https://activecalculus.org/multi/S-10-8-Lagrange-Multipliers.html#S-10-8-Lagrange-Multipliers-6-14}{AC Multi 10.8 Exercise 14c}) Find the absolute maximum and minimum of $f(x,y,z) = x^2+y^2+z^2$ subject to the constraint that $(x-3)^2 + (y+2)^2 + (z-5)^2 \le 16\text{.}$ (Hint: here the constraint is a closed, bounded region. Use the boundary of that region for applying Lagrange Multipliers, but don't forget to also test any critical values of the function that lie in the interior of the region.)

\begin{red}
Let's do the easy part first and check for critical points inside the region. Since $\del f = \langle 2x, 2y, 2z \rangle$, the only critical point is $(0, 0, 0)$. This isn't inside the region because $(0-3)^2 + (0+2)^2 + (0-5)^2 = 9 + 4 + 25 = 38$, which is certainly bigger than 16. So we can ignore that critical point; any maxes or mins in this situation will occur on the boundary.

Let's set up our Lagrange multiplier equations $\del f = \lambda \del g$ together with the constraint:
\begin{align*}
    2x &= \lambda\cdot 2(x-3) \\
    2y &= \lambda \cdot 2(y+2) \\
    2z &= \lambda \cdot 2(z-5) \\
    (x-3)^2 &+ (y+2)^2 + (z-5)^2 = 16
\end{align*}
I asked W$|$A to solve this system of equations so I could skip a lot of tedious algebra. (I used $L$ for $\lambda$ because I can't type $\lambda$ into W$|$A.) Here's the results:
\begin{align*}
    \lambda &= 1 -\frac{1}{2}\sqrt{\frac{19}{2}} &
    x &= 3 - 6 \sqrt{\frac{2}{19}} &
    y &= -2 + 4 \sqrt{\frac{2}{19}} &
    z &= 5 - 10 \sqrt{\frac{2}{19}} \\
    &\approx -0.541104 &
    &\approx 1.05335 &
    &\approx -0.702229 &
    &\approx 1.75557 \\
    &&&&&&&\\
    \lambda &= 1 +\frac{1}{2}\sqrt{\frac{19}{2}} &
    x &= 3 + 6 \sqrt{\frac{2}{19}} &
    y &= -2 - 4 \sqrt{\frac{2}{19}} &
    z &= 5 + 10 \sqrt{\frac{2}{19}} \\
    &\approx 2.5411 &
    &\approx 4.94666 &
    &\approx -3.29777 &
    &\approx 8.24443
\end{align*}

One of these is going to give us a min and the other will give us the max. To tell which is which, we just need to feed our points into $f(x,y,z)$:
\begin{align*}
    f(1.05334, -0.702229, 1.75557) &= 4.68468 \\
    f(4.94666, -3.29777, 8.24443) &= 103.315
\end{align*}

Cool, so the first point definitely is the location of the minimum, and the second is definitely the location of the maximum.

(By the way, a geometric interpretation of this problem: $f(x,y,z)$ is the square of the distance from your point $(x,y,z)$ to the origin, and the region is a sphere centered at $(3, -2, 5)$ with radius 4, together with the inside of the sphere. So you're looking for the minimum and maximum (square of the) distance from the sphere to the origin. Since the origin isn't actually inside this sphere (check!), it makes sense that both the min and the max are on the boundary. You can even check that these two points are ``antipodal.'')
\end{red}

\end{enumerate}

\end{document}