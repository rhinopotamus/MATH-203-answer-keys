\documentclass[12pt]{article}
\usepackage[margin=1in, paperwidth=8.5in, paperheight=11in]{geometry}
\usepackage{ifpdf,amsmath, amssymb, comment, color, graphicx, stmaryrd,setspace,tikz, fancyhdr, wrapfig, textcomp, mathptmx, siunitx}

\usepackage[shortlabels]{enumitem}


\usepackage{hyperref}
\hypersetup{
    colorlinks=true,
    urlcolor=blue,
}

\setlength{\headheight}{14.5pt}
\newcommand{\Q}{\mathbb{Q}}
\newcommand{\R}{\mathbb{R}}
\newcommand{\Z}{\mathbb{Z}}
\newcommand{\vu}{\mathbf{u}}
\newcommand{\vv}{\mathbf{v}}
\newcommand{\vw}{\mathbf{w}}
\newcommand{\vi}{\mathbf{i}}
\newcommand{\vj}{\mathbf{j}}
\newcommand{\vk}{\mathbf{k}}
\newcommand{\vn}{\mathbf{n}}
\newcommand{\vr}{\mathbf{r}}
\newcommand{\proj}{\operatorname{proj}}
\newcommand{\orth}{\operatorname{orth}}
\newcommand{\comp}{\operatorname{comp}}
\newcommand\dotp[1][.5]{\,\mathbin{\vcenter{\hbox{\scalebox{#1}{$\bullet$}}}}\,}
	
% Solution text is in red. If you want the solutions to show, remove the \iffalse from the definition of the \red command.
\newenvironment{red}{\color{red}}{\ignorespacesafterend}
\newcommand{\blue}[1]{\textcolor{blue}{#1}}
\newcommand{\green}[1]{\textcolor{green}{#1}}
\renewcommand{\section}[1]{\begin{center} \textbf{#1} \\\end{center}}
%
\hyphenpenalty=5000
\setlength{\parindent}{0in}
%\oddsidemargin=-.25in
\allowdisplaybreaks
\pagestyle{fancy}
\renewcommand{\headrulewidth}{0pt}
\lhead{MATH 203}
\rhead{Fall 2024}
%\lfoot{\copyright\ CLEAR Calculus 2010}
\cfoot{}

\begin{document}
%


%\onehalfspacing
\allowdisplaybreaks
%##################################################################
\section{Checkpoint: Lines and planes }
Consider these three planes:

\begin{itemize}
    \item $p_1$ is given by the scalar equation $-2x + y - 3z = 4$
    \item $p_2$ is the plane through the points $(0, -1, 0)$, $(1,1,0)$, and $(0,2,1)$
    \item $p_3$ has a normal vector $\vn = \langle 1, 2, 1\rangle$ and passes through the point $(4, -2, 1)$
\end{itemize}

\begin{enumerate}[(a)]
    \item Two of these planes are parallel and therefore do not intersect. Which ones? \\(Hint: look at the normal vectors.)
    \item For two of the planes that \textit{do} intersect, write an equation for the line of intersection.
\end{enumerate}




	
\end{document}