\documentclass[10pt]{article}
\usepackage[margin=1in, paperwidth=8.5in, paperheight=11in]{geometry}
\usepackage{ifpdf, amsmath, amssymb, comment, color, graphicx, stmaryrd, setspace, enumitem, fancyhdr, wrapfig, textcomp, mathptmx, siunitx, multicol}
\usepackage{hyperref}
\hypersetup{
    colorlinks=true,
    urlcolor=blue,
}

\usepackage{tikz}
\usetikzlibrary{trees}

\setlength{\headheight}{14.5pt}
\newcommand{\del}{\nabla}
\newcommand{\Q}{\mathbb{Q}}
\newcommand{\R}{\mathbb{R}}
\newcommand{\Z}{\mathbb{Z}}
\newcommand{\vu}{\mathbf{u}}
\newcommand{\vv}{\mathbf{v}}
\newcommand{\vw}{\mathbf{w}}
\newcommand{\vi}{\mathbf{i}}
\newcommand{\vj}{\mathbf{j}}
\newcommand{\vk}{\mathbf{k}}
\newcommand{\vn}{\mathbf{n}}
\newcommand{\vr}{\mathbf{r}}
\newcommand{\vs}{\mathbf{s}}
\newcommand{\va}{\mathbf{a}}
\newcommand{\vF}{\mathbf{F}}
\newcommand{\vL}{\mathbf{L}}
\newcommand{\vT}{\mathbf{T}}
\newcommand{\vN}{\mathbf{N}}
\newcommand{\vB}{\mathbf{B}}
\newcommand{\comp}{\operatorname{comp}}
\newcommand{\proj}{\operatorname{proj}}
\newcommand{\orth}{\operatorname{orth}}
\newcommand\dotp[1][.5]{\,\mathbin{\vcenter{\hbox{\scalebox{#1}{$\bullet$}}}}\,}


\newenvironment{red}{\color{red}}{\ignorespacesafterend}
\newcommand{\blue}[1]{\textcolor{blue}{#1}}
\newcommand{\green}[1]{\textcolor{green}{#1}}
\renewcommand{\section}[1]{\begin{center} \textbf{#1} \\\end{center}}
%
\hyphenpenalty=5000
\setlength{\parindent}{0in}
%\oddsidemargin=-.25in
\allowdisplaybreaks
\pagestyle{fancy}
\renewcommand{\headrulewidth}{0pt}
\lhead{MATH 203}
\rhead{Fall 2024}
%\lfoot{}
%\cfoot{}

\begin{document}
%


%\onehalfspacing
\allowdisplaybreaks
%##################################################################
\section{Checkpoint: Partial derivatives}
The gas law for a fixed mass $m$ of an ideal gas at absolute temperature $T$ in Kelvins, pressure $P$ in pascals, and volume $V$ in liters is $PV = mRT$, where $R$ is the gas constant.

\begin{itemize}
    \item Compute each of the partial derivatives below,
    \item assign them the correct units,
    \item and say a sentence or two about what each one means. (Be careful to think about what is \textit{constant} and what is \textit{changing}.)
\end{itemize}

\begin{enumerate}[(a)]
    \item $\dfrac{\partial P}{\partial V}$
    \item $\dfrac{\partial V}{\partial T}$
    \item $\dfrac{\partial T}{\partial P}$
\end{enumerate}

(Bonus question for the physics-knowers: Why does the sign of each one make sense?)
	
\end{document}