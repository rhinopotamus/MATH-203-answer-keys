\input{../header}

\begin{document}
%


%\onehalfspacing
\allowdisplaybreaks
%##################################################################
\section{Checkpoint: Partial derivatives}
The gas law for a fixed mass $m$ of an ideal gas at absolute temperature $T$ in Kelvins, pressure $P$ in pascals, and volume $V$ in liters is $PV = mRT$, where $R$ is the gas constant.

\begin{itemize}
    \item Compute each of the partial derivatives below,
    \item assign them the correct units,
    \item and say a sentence or two about what each one means. (Be careful to think about what is \textit{constant} and what is \textit{changing}.)
\end{itemize}

\begin{enumerate}[(a)]
    \item $\dfrac{\partial P}{\partial V}$
    \item $\dfrac{\partial V}{\partial T}$
    \item $\dfrac{\partial T}{\partial P}$
\end{enumerate}

(Bonus question for the physics-knowers: Why does the sign of each one make sense?)
	
\end{document}